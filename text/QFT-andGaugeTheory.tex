\documentclass[../main.tex]{subfiles} % Due to use of package subfiles

%%%%%%%%%%%%%%%%%%%%%%%%%%%%%%%%%%%%%%%%%%%%%%%%%%%%%%%%%%%%%%%%%%%%%%%%%%%%%%%%

\begin{document}

\chapter{Quantum Field Theory and Gauge~Theory} \label{chap:ContinuumQFT}

\ldots




\section{Relativistic tensor notation}

In this section the notation of relativistic tensors is presented following Refs. \cite{peskin_introToQFT_1995,Uggerhoej_SpecielRelativitetsteori_2016}. A \emph{four-vector}\index{four-vector}, which is a tensor of rank 1, is a vector with four entries denoting the coordinates of space-time, $a = (a_0,\, a_1,\, a_2,\, a_3)$, with index $0$ denoting the temporal coordinate and index $i$ ($i = 1,2,3$) denotes the spatial coordinates, thus the four-vector might be written as $a = (a_0,\, \Vec{a})$ for $\Vec{a}$ being the regular spatial three-vector. We will be using indices to refer to specific entries of the four-vector\footnote{Indices will also be used to denote the entries of tensors of higher ranks. The number of indices corresponds to the rank of the tensor.}; Greek letter indices run over $0,1,2,3$, while Roman letter indices only run over the spatial coordinates i.e. $1,2,3$.

In line with the use of indices, we will introduce and employ the Einstein summation convention also known as \emph{Einstein notation}\index{Einstein notation}. In this convention repeated indices indicates an implicit summation over these indices, thus
\begin{align} \label{eq:EinsteinSummation}
    a^\mu b_\mu &\equiv \sum_\mu a^\mu b_\mu \: .
\end{align}
One might notice, that the indices in \cref{eq:EinsteinSummation} is placed either as a sub- or a superscript. This is due to a tensor\footnote{Tensors of rank 2 or higher is often denoted with a capital letter, thus we here use the capital letter instead of the otherwise non-capital letters used for the four-vectors.} being either contravariant, $A^\mu$, or covariant, $A_\mu$. For each contravariant four-vector $a^\mu = (a_0,\, \Vec{a})$ its corresponding covariant four-vector can be calculated as
\begin{align} \label{eq:CovariantFromContravariant}
    a_\mu &= g_{\mu\nu}a^\nu = (a_0,\, -\Vec{a}) \: ,
\end{align}
where the last equality in \cref{eq:CovariantFromContravariant} is due to our choise of the rank 2 metric tensor of spacetime to be the Minkowski metric tensor, $g_{\mu\nu} = \text{Diag}(1,\, -1,\, -1,\, -1)$, since we will be working in a flat space-time.

Some important four-vectors that will be used later on is the space-time coordinate vector $x^\mu = (t, \Vec{x})$, which is contravariant, and the covariant derivative\index{covariant derivative}
\begin{align}
    \partial_\mu &= \pdv{}{x^\mu} = \left(\pdv{}{x^0},\, \Vec{\nabla}\right) = \left(\partial_t,\, \Vec{\nabla}\right) \: .
\end{align}
Note that even thought the derivative is covariant, we are actually differentiating with respect to the contravariant space-time coordinate vector.

We will be using this notation due to it being neat when expressing entrywise equality of equations, and due to it securing covariance of the equations, thus the equations will be independent of the reference frame used \cite{Uggerhoej_SpecielRelativitetsteori_2016}, which is important, since this is the fundamental principle of relativity.


% g_µ\nu metric tensor of the Minkowski space




% Four-vector, Einstein notation/summation, Minkowski metric tensor, covariant differential-vector




\section{Lagrangian formalism of fields}

In quantum field theory the dynamics of a system is described using the Lagrangian formalism \cite{peskin_introToQFT_1995,Clark_LagrangianQFT_1988}, but instead of considering the Lagrangian itself one consider the \emph{Lagrangian density}\index{Lagrangian density} $\L$, which is a function of one or more fields \emph{fields}\index{field} $\{\psi_i\}$ and their space- and time-derivatives $\{\partial_\mu \psi_i\}$. From this the action can be calculated as
\begin{align}
    S &= \int \dd{t} L = \int \dd{t} \int \dd[d]{x} \L = \int \dd[d+1]{x} \L \: ,
\end{align}
with $L$ being the Lagrangian and $d$ being the number of spatial dimensions. The action is being presented due to it being defining of symmetries, as we will use in \cref{sec:ContinuumQFT_LocalU(1)GaugeInvariance}.

The Lagrangian formulation is widely used when working with relativistic dynamics, due to the expressions all being explicitly Lorentz invariant \cite{peskin_introToQFT_1995}, hence we will be using this formulation to reach the correct invariance (\cref{sec:ContinuumQFT_LocalU(1)GaugeInvariance}), but due to us working with quantum mechanics, we are going to switch to the Hamiltonian formalism in \cref{sec:ContinuumSchwingerModelHamiltonianDensity} instead. For this for this formalism one first needs to set up a Lagrangian density, which can then be used to calculate the \emph{conjugate momentum fields}\index{conjugate momentum field} \cite{peskin_introToQFT_1995,Clark_LagrangianQFT_1988}
\begin{align} \label{eq:LagrangianFormalism_ConjugateMomentumField}
    \pi_i &= \pdv{\L}{\dot{\psi}_i} \: ,
\end{align}
for which $\dot{\psi}_i = \partial_0\psi_i$. To turn the equations quantum, the fields and conjugate momentum fields is changed to operators, by recasting them in terms of fermionic (or bosonic) creations and annihilation operators \cite{panyella_masterThesis_2019}.

Now the \emph{Hamiltonian density}\index{Hamiltonian density} corresponding to the Lagrangian density can be found using
\begin{align} \label{eq:LagrangianFormalism_HamiltonianDensity}
    \H &= \sum_i \pi_i \dot{\psi}_i - \L \: ,
\end{align}
and from this the \emph{Hamiltonian}\index{Hamiltonian} is calculated as
\begin{align} \label{eq:LagrangianFormalism_Hamiltonian}
    H &= \int \dd[d]{x} \H \: ,
\end{align}
again with $d$ being the number of spatial dimensions \cite{Clark_LagrangianQFT_1988}.




\index{Schwinger model!continuum!derivation|(}
\section{The continuum Schwinger model} \label{Sec:ContinuumSchwingerModel}

% Most of the content of this section can be found in Refs. \cite{peskin_introToQFT_1995,panyella_masterThesis_2019}.



\subsection{Dirac field}

Due to our interest in examining confinement of quarks, which are spin-\half fermions, we will take as the starting point the most prominent fermionic field, namely the Dirac field. For this field the Lagrangian density\index{Lagrangian density!of Dirac field} is
\begin{align} \label{eq:DiracLagrangiangDensity}
    \L_D &= \psibar (i \gamma^\mu \partial_\mu - m) \psi \: ,
\end{align}
where $\psibar = \gamma^0 \psi\dagger$, $\psi$ is a \emph{Dirac spinor}\index{spinor}\index{spinor!Dirac}\index{Dirac spinor|see{spinor, Dirac}}, and $\gamma^\mu$ are the gamma- or \emph{Dirac-matrices}\index{Dirac matrices}\index{gamma-matrices|see{Dirac-matrices}} \cite{peskin_introToQFT_1995,sakurai_modernQM_2017}. For the following descriptions of the above mentioned elements Ref. \cite{peskin_introToQFT_1995} will be laying the foundation. The Dirac spinor $\psi$\index{Spinor!Dirac} is a four component spinor, with each entry being complex. The upper two entries of $\psi$ corresponds to a particle, while the bottom two corresponds to an antiparticle, and for each of these pairs, the top and bottom entries represent the two possible spin states of the spin-\half particle or antiparticle. Now to defining the Dirac-matrices $\gamma^\mu = (\gamma^0,\, \gamma^i)$ for $i=1,2,3$. These matrices are $4 \times 4$ constant matrices that satisfy the Clifford algebra, thus the anticommutator
\begin{align}
    \anticommutator{\gamma^\mu}{\gamma^\nu} &= 2g^{\mu\nu}\id_4 \: ,
\end{align}
with $g^{\mu\nu}$ being the Minkowski metric and $\id_4$ a $4 \times 4$ identity matrix. In $2 \times 2$ block form the Dirac-matrices is
\begin{align}
    \gamma^0 &=
        \begin{pmatrix}
            \id_2 & 0 \\
            0 & -\id_2
        \end{pmatrix} \: , \qquad
    \gamma^i =
        \begin{pmatrix}
            0 & \sigma^i \\
            -\sigma^i & 0
        \end{pmatrix} \: ,
\end{align}
with $\id_2$ being a $2 \times 2$ identity matrix and $\sigma^i$ being the Pauli matrices.

As we later will restrict ourselves to only working in (1+1)-dimension, one spatial and one temporal dimension (often denoted (1+1)D), the Dirac-matrices \ldots



\subsection{Local U(1) gauge invariance} \label{sec:ContinuumQFT_LocalU(1)GaugeInvariance}

When considering systems one may often also consider their symmetries. Generally a \emph{symmetry}\index{symmetry} is written as an invariance of the action, and thus of the Lagrangian density, with respect to transformation represented by operators \cite{peskin_introToQFT_1995}. The set of these operators is called the \emph{symmetry group}\index{symmetry!symmetry group} due to the Lagrangian density being invariant under these and their compositions \cite{panyella_masterThesis_2019}. In the following we will consider the unitary gauge U(1)\index{U(1)}\index{symmetry!U(1)|see{U(1)}}, where the name refers to it being the group of unitary $1\times1$ matrices also known complex numbers with modulus $1$ \cite{peskin_introToQFT_1995}. One of these transformations is the phase of a wave function
\begin{align} \label{eq:GlobalU(1)GaugeInvariance}
    \psi &\rightarrow U_\alpha \psi = \exp{i\alpha}\psi \: ,
\end{align}
with $\alpha$ being a real number. A Lagrangian density being invariant under this transformation is said to be \emph{globally gauge invariant}\index{gauge invariance!global}, thus the wave function's phase will vanish by choosing the correct gauge. Due to $U_\alpha$ being a number and exploiting its unitarity, $U_\alpha\dagger U_\alpha = \id$, it is trivial to show that the Dirac Lagrangian density $\L_D$ is globally U(1) gauge invariant.

In developing quantum field theory one shall not only impose global gauge invariance but instead demand the more strict \emph{local gauge invariance}\index{gauge invariance!local} \cite{griffiths_introToElementaryParticles_2008}. For the transformation in \cref{eq:GlobalU(1)GaugeInvariance} this implies, that the parameter $\alpha$ shall be spacetime-dependent, $\alpha(x^\mu)$. As for most systems, the Dirac Lagrangian density is not locally invariant as it is, thus introduction of new fields for the system to interact with is required \cite{panyella_masterThesis_2019}. As it happens, the requirement of local gauge invariance supplied a systematic way for determining the equations describing the fundamental interactions: The electromagnetic interaction requires -- as we will derive later in this section -- local U(1) invariance, the electroweak interaction requires local SU(2)$\otimes$U(1) invariance\footnote{SU(N)\index{symmetry!SU(N)} being the the group of all unitary $N \times N$ matrices with determinant $1$ \cite{peskin_introToQFT_1995}.}, and the strong nuclear interaction requires local SU(3) invariance \cite{stanford_QFT, griffiths_introToElementaryParticles_2008}, thus the Standard Model of elementary particle physics is a gauge theory resulting from demanding local U(1)$\otimes$SU(2)$\otimes$SU(3) symmetry \cite{stanford_historyOfQFT}.

In the following it will be shown, that the Dirac Lagrangian density, \cref{eq:DiracLagrangiangDensity}, is not locally U(1) invariant, but it will lay the foundation for deriving one that is. The derivation will follow that of Refs. \cite{peskin_introToQFT_1995, griffiths_introToElementaryParticles_2008, panyella_masterThesis_2019}. To lighten the the notation, the parameter $\alpha(x^\mu)$ will be denoted $\alpha$ for the rest of this section. From \cref{eq:GlobalU(1)GaugeInvariance} the conjugated field must transform as
\begin{align}
    \psibar &\rightarrow \gamma^0 (\psi')\dagger
        = \gamma^0 (U_\alpha \psi)\dagger
        = \gamma^0\psi\dagger U_\alpha\dagger
        = \psibar\exp{-i\alpha} \: ,
\end{align}
for $\psi'$ being the U(1) transformed field, $\psi' = U_\alpha \psi$. Thus Dirac's Lagrangian density will transform as
\begin{align} \label{eq:DiracLagrangianDensityLackLocalU(1)Symmetry}
\begin{split}
    \L_D = \psibar (i \gamma^\mu \partial_\mu - m) \psi \rightarrow
        & \psibar \exp{-i\alpha} (i \gamma^\mu \partial_\mu - m) \exp{i\alpha} \psi \\
        =& \psibar \exp{-i\alpha} \exp{i\alpha} (i \gamma^\mu [i \partial_\mu \alpha + \partial_\mu] - m) \psi \\
        =& \L_D - \psibar\gamma^\mu \partial_\mu \alpha \psi
\end{split}
\end{align}
exploiting again the unitarity of the U(1) symmetry group, $U_\alpha\dagger U_\alpha = \id$. From \cref{eq:DiracLagrangianDensityLackLocalU(1)Symmetry} it can be seen, that the differential operator is to blame for the lack of local gauge invariance. Thus one must introduce the \emph{gauge covariant derivative}\index{gauge covariant derivative}
\begin{align} \label{eq:GaugeCovariantDerivative}
    D_\mu &= \partial_\mu + iqA_\mu \: ,
\end{align}
which will replace the original derivative. In \cref{eq:GaugeCovariantDerivative} $A_\mu$\index{Amu@$A_\mu$} is a vector field which under U(1) symmetry transforms as
\begin{align}
    qA_\mu &\rightarrow qA_\mu - \partial_\mu\alpha \: ,
\end{align}
for $q$ being a real number. Using this new gauge covariant derivative, the Lagrangian density becomes locally U(1) gauge invariant:
\begin{align}
\begin{split}
    \psibar (i \gamma^\mu D_\mu - m) \psi \rightarrow
        & \psibar \exp{-i\alpha} (i \gamma^\mu [\partial_\mu + iqA_\mu - i\partial_\mu\alpha] - m) \exp{i\alpha} \psi \\
        =& \psibar (i \gamma^\mu [\partial_\mu + i \partial_\mu \alpha + iqA_\mu - i\partial_\mu\alpha] - m) \psi \\
        =& \psibar (i \gamma^\mu D_\mu - m) \psi \: .
\end{split}
\end{align}
However, there is still one important addition to the Lagrangian density, which we have still not accounted for: For the newly introduced vector field we have only introduced its coupling to the already existing field, $\psi$, but we also need to include its own ''free'' term, i.e. its kinetic energy term \cite{griffiths_introToElementaryParticles_2008}. The usual form of the kinetic energy is to be a scalar and the square of the derivative of the field, thus we set these as the requirements of $A_\mu$'s kinetic energy. Thus we propose the term $F_{\mu\nu}F^{\mu\nu}$,
\begin{align}
    F_{\mu\nu} = \partial_\mu A_\nu - \partial_\nu A_\mu \: ,
\end{align}
being the \emph{gauge field tensor}\index{gauge field tensor}\index{Fmunu@$F_{\mu\nu}$|see{gauge field tensor}} \cite{griffiths_introToEldyn_2017}. Since the gauge field tensor transforms as
\begin{align}
    F_{\mu\nu} \rightarrow F_{\mu\nu} - \partial_\mu A_\nu \alpha + \partial_\nu A_\mu \alpha \: ,
\end{align}
it is gauge invariant, assuming that $\alpha$ is well-behaved such that the order of the derivatives is irrelevant, $\partial_\mu\partial_\nu\alpha = \partial_\nu\partial_\mu\alpha$, thus they will cancel each other \cite{panyella_masterThesis_2019}.

Combining the above results in the locally U(1) gauge invariant Lagrangian density
\begin{align} \label{eq:ContinuumSchwingerModelLagrangianDensity}
    \L &= \psibar (i \gamma^\mu D_\mu - m) \psi - \frac{1}{4}F_{\mu\nu}F^{\mu\nu} \: ,
\end{align}
where the factor of $-\nicefrac{1}{4}$ is due to convention. This is the Lagrangian density of \emph{quantum electrodynamics}\index{Lagrangian density!of quantum electrodynamics}, and for (1+1)D it is known as the Lagrangian density of the \emph{continuum Schwinger model}\index{Schwinger model!continuum}\index{Lagrangian density!of Schwinger model!continuum} \cite{melnikov_latticeSchwingerModel_2000}.



\subsection{Hamiltonian density} \label{sec:ContinuumSchwingerModelHamiltonianDensity}

In this section the arguments will mostly follow that of Ref. \cite{melnikov_latticeSchwingerModel_2000}. We choose make the common gauge restriction to work in the temporal gauge, thus the temporal part of the the vector field $A_\mu$ is zero, $A_0 = 0$. From this choice of gauge and that of working in (1+1)-dimensions, i.e. $\mu,\nu = 0,1$, one can reduce \cref{eq:ContinuumSchwingerModelLagrangianDensity} to
\begin{align} \label{eq:ContinuumSchwingerModelLagrangianDensityAfterA0=0}
    \L &= \psibar (i \gamma^\mu \partial_\mu - \gamma^1 q A_1 - m) \psi + \frac{1}{2}E_1E^1 \: ,
\end{align}
where $E_1 = \partial_0 A_1$ is the electric field in the direction of $x_1$. Computing the last term of \cref{eq:ContinuumSchwingerModelLagrangianDensityAfterA0=0} requires one to remember, that the gauge field tensor is an antisymmetric tensor, $F_{\mu\nu} = - F_{\nu\mu}$, and that a contravariant tensor can be found by flipping the sign on the entries in the covariant tensor, for which the index is $0$, $F^{0\nu} = -F_{0\nu}$, \cite{griffiths_introToEldyn_2017}, thus
\begin{align}
\begin{split}
    F_{\mu\nu}F^{\mu\nu} &= F_{00}F^{00} + F_{01}F^{01} + F_{10}F^{10} + F_{11}F^{11}     = F_{01}F^{01} + F_{10}F^{10} \\
        &= 2F_{01}F^{01}
        = -2F_{01}^2
        = -2(\partial_0A_1 - \partial_1A_0)^2
        = -2(\partial_0A_1)^2 \: ,
\end{split}
\end{align}
for $\mu,\nu = 0,1$.

The continuum Schwinger model Hamiltonian density can now be found using \cref{eq:LagrangianFormalism_ConjugateMomentumField,eq:LagrangianFormalism_HamiltonianDensity,eq:LagrangianFormalism_Hamiltonian} on \cref{eq:ContinuumSchwingerModelLagrangianDensityAfterA0=0}. The Hamiltonian density\index{Schwinger model!continuum}\index{Hamiltonian density!of Schwinger model!continuum} therefor becomes
\begin{align} \label{eq:ContinuumSchwingerModelHamiltonianDensity}
\begin{split}
    \H &= \pdv{\L}{(\partial_0 A_1)} \partial_0 A_1 + \pdv{\L}{(\partial_0 \psi)} \partial_0 \psi - \L \\
        &= \partial^0 A^1 \partial_0 A_1 + \psibar i \gamma^0 \partial_0 \psi - \psibar (i \gamma^\mu \partial_\mu - \gamma^1 q A_1 - m) \psi - \frac{1}{2} E_1 E^1 \\
        &= E^1 E_1 + \psibar i \gamma^0 \partial_0 \psi - \psibar i \gamma^0 \partial_0 \psi - \psibar (i \gamma^1 \partial_1 - \gamma^1 q A_1 - m) \psi - \frac{1}{2} E_1 E^1 \\
        &= - \psibar (i \gamma^1 D_1 - m) \psi + \frac{1}{2} E_1 E^1 \: ,
\end{split}
\end{align}
and thus we get the Hamiltonian\index{Hamiltonian!of Schwinger model!continuum} as the spatial integral of the above density
\begin{align}
    H &= \int \dd{x} \left[ - \psibar (i \gamma^1 D_1 - m) \psi + \frac{1}{2} E_1 E^1 \right] \: .
\end{align}
\index{Schwinger model!continuum!derivation|)}



\end{document}