\documentclass[../main.tex]{subfiles} % Due to use of package subfiles

%%%%%%%%%%%%%%%%%%%%%%%%%%%%%%%%%%%%%%%%%%%%%%%%%%%%%%%%%%%%%%%%%%%%%%%%%%%%%%%%

\begin{document}

\chapter{Introduction} \label{chap:Introduction}

% About QFT
Quantum field theory is a theoretical framework extending quantum mechanics by combining it with special relativity and classical field theory. While quantum mechanics is concerned with describing one or a few particles and relativistic quantum mechanics extends this focus to include the spin of the particles, quantum field theory is a theory enabling us to describe systems of many particles and also allows the treatment of fields, thus both particles and fields can be treated in the same framework. \cite{stanford_QFT}

% History of QFT (development of QED and QCD)
\ldots

% Standard model -> Confinement and QCD
Today one of the most successful theories in physics is the so called \emph{Standard Model}\index{Standard Model} of elementary particle physics \cite{peskin_introToQFT_1995} describing three of the four fundamental forces: The strong nuclear force, the weak nuclear force and the electromagnetic force, but not the gravitational force. As it turns out the Standard Model is a gauge theory resulting from the demand of local U(1)$\otimes$SU(2)$\otimes$SU(3) gauge invariance (see \cref{sec:ContinuumQFT_LocalU(1)GaugeInvariance}) \cite{peskin_introToQFT_1995, stanford_historyOfQFT}.
One of the beautiful key mechanisms of the Standard Model is the confinement of colour charge, also known as \emph{colour confinement}\index{colour confinement} or just confinement, which states that colourful particles as quarks cannot exist freely but only exist in hadrons \cite{peskin_introToQFT_1995, wilson_confinement_1974, griffiths_introToElementaryParticles_2008}.

%Schwinger model (as toy model)
\ldots

% My project
My project have been concerning itself with the understanding of the theoretical framework of quantum field theory, and thus gauge theories, both in the continuum and on a lattice. In \cref{chap:ContinuumQFT} I take as a starting point Dirac's fermionic Lagrangian density and impose local U(1) gauge invariance leading to the derivation of quantum electrodynamics, and thus the continuum Schwinger model. In \cref{chap:LatticeQFT} the continuum Schwinger model is discretized using lattice gauge theory and then approximated by spin matrices, such that it can be modelled on a computer by a well known system. % Lastly \cref{chap:Confinement} serves as an outlook

% Natural units
Lastly it shall be mentioned that throughout this project we will be using the natural units $\hbar = c = 1$, thus length and time has the same unit and this being the reciprocal unit of energy and mass, $[\mathrm{length}] = [\mathrm{time}] = [\mathrm{energy}]^{-1} = [\mathrm{mass}]^{-1}$. This choise is due to it being convention when working with field theory, since we then are able to express most dimensionful quantities using only a single unit, and it simplifies our equations that we do not have to write the constants in every equation.


\end{document}